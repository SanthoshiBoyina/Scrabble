\documentclass[13.5pt]{beamer}

\mode<presentation>
{
  \usetheme{default}     
  \usecolortheme{default} 
  \usefonttheme{serif}  
  \setbeamertemplate{navigation symbols}{}
  \setbeamertemplate{caption}[numbered]
} 

\usepackage[english]{babel}
\usepackage[utf8x]{inputenc}
\usepackage{multicol}

\title[SCRABBLE]{SCRABBLE}
\author{SHRI VISHNU ENGINEERING COLLEGE FOR WOMEN}
\institute{BHIMAVARAM}


\begin{document}

\begin{frame}
  \titlepage
\end{frame}

\begin{frame}{BATCH - 3}
\begin{enumerate}
    \item BATCHU VEDA LIKITHA \hspace {0.5cm}19B01A1209 \hspace {1cm}IT
    \item SHAIK RAHEEMA \hspace{1.7cm} 19B01A05F7 \hspace{0.8cm}  CSE
    \item BOYINA SANTHOSHI \hspace{1.2cm}19B01A0528 \hspace{0.95cm}CSE
    \item VILLA AMRUTHA \hspace{1.7cm} 19B01A04I9 \hspace{1cm}ECE
    \item PASALA YASASVINI \hspace{1.4cm}19B01A02B2 \hspace{0.9cm}EEE
    \item MUTYALA RENUKA SAI \hspace{0.6cm}19B01A0336 \hspace{1cm}MEC

\end{enumerate}
    
    
\end{frame}
\begin{frame}{PROBLEM STATEMENT}
Scrabble is a word game in which two to four players score points by placing tiles, each bearing a single letter, onto a game board divided into a 15 by 15 grid of squares.

\end{frame}

\section{Results}

\begin{frame}

\begin{figure}[htp]
    \centering
    \includegraphics[width=9cm]{scrabble.jpg}
    \caption{An image of SCRABBLE GAME}
    \label{fig:scrabble}
\end{figure}

\end{frame}

\begin{frame}{APPROACH}

\begin{itemize}

\item Built code for tracking of players details.

\item Next on design of game board.

\item Calculated score values.

\item Initialized tiles into bag.

\item Checking of each word whether it is present in dictionary or not.

\item Placed letters on respective positions.


\end{itemize}

\end{frame}

\begin{frame}{PROGRESS}

\begin{itemize}
    \item Day - 1 : \hspace{10cm}
    Amrutha, Raheema, Santhoshi - Designed  game board. Likitha, Renuka, Yasasvini \hspace{0.5cm} -  Initialized score values.
    \item Day - 2 : \hspace{10cm}
    Santhoshi, Renuka, Amrutha \hspace{0.07cm} - Done with tracking of  player details. \hspace{10cm}
    Likitha, Yasasvini, Raheema \hspace{0.1cm} - Implemented turn of  chances.
    \item Day - 3 : \hspace{10cm}
    Likitha, Amrutha, Santhoshi \hspace{0.07cm} - Placing words on board. \hspace{6cm}
    Raheema, Yasasvini, Renuka \hspace{0.07cm} - Checking words whether valid or not.
    
\end{itemize}
    
\end{frame}

\begin{frame}
\begin{itemize}
    \item Day - 4 : \hspace{10cm}
    Likitha, Raheema, Renuka \hspace{0.7cm} - Updating score values.
    Santhoshi, Yasasvini, Amrutha \hspace{0.001cm} - Game for multi user.
    \item Day - 5 : Implemented code to declare winner.
\end{itemize}
    
\end{frame}


\begin{frame}{CHALLENGES}
    \begin{itemize}
    
        \item We faced challenges while creating game for many players.
        
        \item We faced troubles while placing words on the board.
        
        \item To calculate score of player  when he places tiles on premium squares like DLS, DWS, TLS, TWS.
        
        \item Faced difficulty to print Scrabble board on the screen

        \item Faced difficulty to check the word entered by player is formed from given input letters or not
        
    \end{itemize}
\end{frame}


\begin{frame}{LEARNINGS}

\begin{itemize}

\item We gained knowledge on how to work as a team through virtually.

\item Learnt how to make presentations in LaTeX.

\item We attained how to work on project with Python language.

\item We learnt through GitLab how to push files into repository.

\item We have browsed for dictionary file from which we can get atmost valid words.

\end{itemize}
   
\end{frame}

\begin{frame}{TECH STACKS}

\begin{itemize}

\item We have used GitLab to store our files in Repository.

\item We have used Git bash to write commands to store files in GitLab.

\item Python language of (3.9.4 (64 bit) version to build our code as in PyCharm editor.

\item Latex as a extension in Visual Studio Code to prepare presentations.

\item Done whole project in Windows Operating System.
\end{itemize}
   
\end{frame}


\begin{frame}{CODE STACKS}
    \begin{itemize}
        \item We used some conditional statements (if else, else if ladder) and loops (while, for).
        \item We used global variables so as to not to call every variable every time.
        \item We uploaded dictionary.txt, tiles.txt, scrabble.tex, scrabble.txt, scrabble.pdf files. 
    \end{itemize}
\end{frame}

\begin{frame}{STATISTICS}
\begin{itemize}
    \item Totally we used 9 functions such as, game board
, player details, valid word, valid check, word from letters, board display, game play, score, update score.
\item Our code consists of 185 lines.
\item Commits
\begin{figure}[htp]
    \centering
    \includegraphics[width=9cm]{commits.png}
    \label{fig:commits}
\end{figure}
\end{itemize}
    
\end{frame}

\begin{frame}{REFERENCE}
     \begin{itemize}
         \item GitLab link
     \end{itemize}                       
\url{https://gitlab.com/tech__warriors/scrabble}
   
\end{frame}

\begin{frame}
\begin{itemize}
\item Refrence links
    \item \url{https://www.scrapmaker.com/download/data/wordlists/dictionaries/dictionary.txt}
\item \url{https://cdn.download-free-games.com/cf/images/nfe/uploads/board1.jpg}
\item \url{https://www.lucidchart.com/techblog/2016/12/07/how-to-make-a-presentation-in-latex/}
\item \url{https://scrabble-go.en.uptodown.com/android}
\end{itemize}



\end{frame}

\begin{frame}{FUTURE SCOPE}

    We can implement through GUI
\end{frame}

\begin{frame}{BY TECH WARRIORS}

\begin{figure}[htp]
    \centering
    \includegraphics[width=9cm]{Team members.jpeg}
    \label{fig:Team members}
\end{figure}


\end{frame}
\begin{frame}

\centerline{THANK YOU !!!}

\end{frame}

\end{document}
